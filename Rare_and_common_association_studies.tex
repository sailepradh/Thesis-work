## For doing the rare and common variant association study, we first used the master VCF files called as Raw.Vcf 
## For this original file we divided the variaint files into 2 groups called as common and rare vcf files and did the manipulat##ion for the single variant assocaiation for the common variant and subsequenty planned to do the gene based test on the rare ##variant of the gene sets. 

## Part 1: Extracting the VCF files from the orignal raw files
First I made a new folder called Common_rare in the path 

/proj/b2012058/nobackup/private/sailen/

And copied the file raw.vcf into it

cp raw.vcf Common_rare/

## Now we the master VCF files contain samples that were unusable as from the mapping and subsequent preprocessing steps. We ## ##knew that there were some samples that had some problem with either the variants being low thorughput or with the samples are ##out of cluster from IBS plot (Contact Benjamin with the clustering plot). Thus, we only took the samples that had the high ##basecalls and grouped together so as to make their ethinic background same. For this we used the tools VCFtools. The samples ##in consideration were given a file name sample_remove and contains the sample names in the new files as
 
head sample_remove
S0328
S0664
S0580
S0922
S1056

##For the common variants I selected the MAF 0.01 as the base line above which all the variants are selected while for the rare ## or in my opinion to take into the complete mutation profile such that even 1 allele has the variant, I took the MAF as 0.005 ## Now using the vcfools commands we get the following rare and common list variants from the raw vcf master files

 vcftools --vcf raw.vcf --remove sample_remove --remove-filtered-all --maf 0.01 --max-missing 0.95 --recode --out Common
 vcftools --vcf raw.vcf --remove sample_remove --remove-filtered-all --maf 0.005 --max-missing 0.95 --recode --out Rare
 ## THIS command did take a lot of time 
 
 ## Now we have the vcf file as Common.recode.vcf and Rare.recode.vcf and using the wc -l command in the each vcf file I get ##the following count of the vcf files

## Common = 74525 out of a possible 211467 Sites , 211 samples
## Rare = 87477 out of a possible 211467 Sites, 211 sample

## (See log files for the more information about the command and the output)

## Now having the two VCF files we proceed with the Single variant association and rare variant assocaition. For this I thought ## to use single variant association SNPs with the p value < 10-4 and mapping these variants into genes. These genes are then ## tested for the rare varaiant association. Does that make sense???

## Part -2 Associationg of the thrombopenia log nadir value with the variant using plinkseq linear regression model.
 
 module load plinkseq
 
 ## First I wanted to know the preliminary analysis of the individual variant calls: Hence I used the command
 
 pseq Common.recode.vcf i-stats | gcol ID NVAR SING TITV > Common_Individual_stat
 
 ## This showed me that the variant have the Ti/Tv ratio on average of 2.21 which I guess is not good but I will try to work ##with it and lets see if in the further downward process in case/Control there is improvement in the ratio.

##Making the proj files helps in the manipulation and managment of the vcf tools. Hence I created the project file into the ##plinkseq. But for creating the proj we needed the database files f the dbSNP, locus and refseq. So I copied the all the hg ##files and the data files from the previous folder.
 cp -r /Fom_Previous_File/hg19 /proj/b2012058/nobackup/private/sailen/Common_rare
 cp -r /From_Previos_File/data /proj/b2012058/nobackup/private/sailen/Common_rare
 
## However we needed the phenotype information and the phenotypic information was manipulated using R and the unix command  #lines  and stored in the data files such that
 
head data/TPK.phe 
 
##phe1,Float,-9,” Log modified Thrombopenia”
#ID	phe1
S0143	4.42
S0156	3.43
S0160	5.17
S0162	2.64
S0164	1.61
S0170	4.48
S0172	5.49
S0174	4.83
 
## Now making the project file in plinkseq

pseq proj new-project --resources hg19

## loading the file Common.recode.vcf in the plinkseq
pseq proj load-vcf --vcf Common.recode.vcf 
pseq proj tag-file --id 1 --name Common
pseq proj var-summary
pseq proj v-view

## Adding the phenotype information into the project
pseq proj load-pheno --file data/TPK.phe 
pseq proj i-view

## Doing the basic mandatory summary of the locus, database and seqbase

pseq proj loc-summary
pseq proj ref-summary 
pseq proj seq-summary

 pseq proj i-stats > Proj_istats
 
 ## Now after doing the preliminary analysis in plinkseq we wanted to do the linear regression of the quantitative value of the ## thrombopenia. For this we used the command glm in the plinkseq
 
 pseq proj glm --phenotype phe1 > Single_variant
 sort -g -k9 Single_variant > Sorted_Single_variant
 grep -v 'NA' Sorted_Single_variant > Final_Sorted_SNP
 wc -l Final_Sorted_SNP

## This is unadjusted linear regression of the quantitative endophenotype with the variants. We first did with the unajusted of ## the covariates as there may be compounding influence from the covariate. (See the paper "Including known covariates can  reduce #power to detect genetic effects in case-control studies")

#Now we extract the SNPs with the values <10-4

 sed -n "/VAR/,/0.000999356/p" Final_Sorted_SNP > Final_list_SNP
 wc -l Final_List_SNP  ## 89
 
 ## Now to map the Listed SNP to the genes or the near gene( because most of the genes are in the intronic region) I first uploaded the List into the Ensembl tool called as VEP. However here I noticed some discrepancy with the expected result. While slimming through the variant I had seen a gene SERPINC1 in the chromosome 1,which has seen function related to the thrombocytosis. However the ensembl shoed same position to be other gene so ask B about the tools or anything else that could map the variants to the REFSEQ because in the next phase of the association study we have database with the ref database. (See the miscellonous unix commands for the manipulation of the List to get to the final list.txt)


 
